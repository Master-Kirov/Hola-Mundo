\documentclass[12 pt]{beamer}
\usepackage[utf8]{inputenc} 
\usepackage{natbib}
\title{Desarrollo de Vídeojuegos \linebreak Programación y Animación}
\author{15211292 Gallardo Pérez Carlos David}
\date{13 de Abril de 2016}
%Antibes Warsaw
\usetheme{Warsaw}
\setbeamerfont{title}{shape=\itshape}
\setbeamercolor{title}{fg=green!255!black,bg=black!255!white}
\begin{document}
		\titlepage
\begin{center}
		Instituto Tecnológico de Tijuana
		\newline \newline \newline \newline 
\end{center}
		\scriptsize
		
\begin{frame}
		\newpage \newpage \newpage
		\frametitle{Índice}
		%\tableofcontents[pausesections]
		\tableofcontents
\end{frame}
	
\section{Desarrollo de Vídeojuegos}
\subsection{Programación}
\begin{frame}
\begin{center}
\frametitle{Desarrollo de Vídeojuegos}
		JKH
		\cite{DVJ001}
\end{center}
\end{frame}
\begin{definition}
			Un videojuego es concebido como un medio de entretenimiento, en donde se incluye a uno o
varios usuarios, llamados Players o simplemente Jugadores, los cuales mantienen una interacción
constante con varias interfaces, como pueden ser los joysticks o controles, teclado, mouse, entre otros,
y un dispositivo de video (monitor de PC, TV, Realidad Virtual, etc).
		\end{definition}
\bibliographystyle{apalike}
\bibliography{Ref}
\end{document}
